\documentclass[10pt,CCT]{ctexart}
\textheight=23cm \textwidth=16cm \oddsidemargin .5cm \topmargin
-2cm

\usepackage{amsmath}

\setlength{\parindent}{2\ccwd}          %段落首行缩进量 \ccwd表示一个汉定的宽度
\setlength{\parskip}{2pt plus1pt minus1pt}  %段落之间的竖直距离
\setlength{\baselineskip}{20pt plus2pt minus1pt}%段落内的行距
\setlength{\textheight}{22true cm}      %每页上的文本总高度
\setlength{\textwidth}{15true cm}     %每页上的文本总宽度

\newcommand{\beq}{\begin{eqnarray}}
\newcommand{\eeq}{\end{eqnarray}}
\newcommand{\be}{\begin{equation}}
\newcommand{\ee}{\end{equation}}
\newcommand{\bq}{\begin{eqnarray*}}
\newcommand{\eq}{\end{eqnarray*}}

\newcommand{\D}{\scriptscriptstyle}
\newtheorem{lemma}{{\heiti 引理}}
\newtheorem{theorem}{{\heiti 定理}}
\newtheorem {definition}{{\heiti 定义}}
\newtheorem{corollary}{{\heiti 推论}}

\renewcommand{\theequation}{\thesection.\arabic{equation}}
\renewcommand{\thetheorem}{\thesection.\arabic{theorem}}
\renewcommand{\thecorollary}{\thesection.\arabic{corollary}}
\renewcommand{\thedefinition}{\thesection.\arabic{definition}}
\renewcommand{\thelemma}{\thesection.\arabic{lemma}}

%生成标题,即使上面几条标题命令的定义生效

\begin{document}%

\title{
{\heiti《数理逻辑》第 {$1$} 次作业
}
}
\date{\today}

\author{
姓名:\underline{杨骏飞}~~~~~~
学号:\underline{200111427}~~~~~~}


\maketitle

\noindent
{\bf 题目1:} 将下列语句形式化为命题公式
\begin{enumerate}
\item[(1)] 2既是素数又是偶数。
\item[(2)] 一个整数是奇数当且仅当它不能被2整除。
\item[(3)] 大学里的学生不是本科生就是研究生。
\item[(4)] 你的车速超过每小时100公里足以接到超速罚单。
\end{enumerate}

\vspace{5pt}
\noindent
{\bf 答:}
\begin{enumerate}
    \item [(1)] 设$p$:2是素数,$q$:2是偶数;\\化为命题公式:$p \wedge q$ 。
    \item [(2)] 设$p$:一个整数是奇数,$q$:它不能被2整除;\\化为命题公式:$p \leftrightarrow q$ 。
    \item [(3)] 设$p$:大学里的学生是本科生,$q$:这个学生是研究生;\\化为命题公式: $\neg p \leftrightarrow q$ 。
    \item [(4)] 设$p$:你的车速超过每小时100公里,$q$:接到超速罚单;\\化为命题公式: $p \rightarrow q$ 。
\end{enumerate}

\vspace{10pt}
\noindent
{\bf 题目2:}判断下列逻辑蕴含和逻辑等价是否成立
\begin{enumerate}
\item[(1)] $A\Rightarrow B\rightarrow A$
\item[(3)] $A \rightarrow (B \rightarrow C) \Rightarrow (A\rightarrow B) \rightarrow (A \rightarrow C)$
\item[(5)] $(A \vee B) \rightarrow C \Leftrightarrow (A \rightarrow C) \wedge (B\rightarrow C)$
\end{enumerate}

\vspace{5pt}
\noindent
{\bf 答:}
\begin{enumerate}
    \item [(1)] 成立,证明如下\\当$A^{v}=1$时,$(B \rightarrow A)^{v}=1-B^{v}+B^{v}A^{v}=1-B^{v}+B^{v}=1$
    \item [(3)] 成立,证明如下
    \begin{equation}
        \begin{aligned}
            (A \rightarrow (B \rightarrow C))^{v}&=1-A^{v}+A^{v}(B \rightarrow C)^{v}\\
            &=1-A^{v}+A^{v}(1-B^{v}+B^{v}C^{v})=1
        \end{aligned}\notag
    \end{equation}
    即$$A^{v}B^{v}(1-C^{v})=0$$
    所以$$A^{v}=0\mbox{或}B^{v}=0\mbox{或}C^{v}=1$$
    当$A^{v}=0$时,$(A \rightarrow B)^{v}=1$,$(A \rightarrow C)^{v}=1$,故$((A\rightarrow B) \rightarrow (A \rightarrow C))^{v}=1$\\
    当$B^{v}=0$且$A^{v}=1$时,$(A \rightarrow B)^{v}=1-A^{v}=0$,故$((A\rightarrow B) \rightarrow (A \rightarrow C))^{v}=1$\\
    当$C^{v}=1$时,$(A \rightarrow C)^{v}=1$,故$((A\rightarrow B) \rightarrow (A \rightarrow C))^{v}=1$\\
    综上所述,命题成立
    \item [(5)] 成立,证明如下
    \begin{equation}
        \begin{aligned}
            Left &= ((A \vee B) \rightarrow C)^{v}=1-(A \vee B)^{v}+(A \vee B)^{v}C^{v}\\
            &= 1-A^{v}-B^{v}+A^{v}B^{v}+A^{v}C^{v}+B^{v}C^{v}-A^{v}B^{v}C^{v}\\
            Right &= (1-A^{v}+A^{v}C^{v}) \vee (1-B^{v}+B^{v}C^{v})\\
            &= 1-B^{v}+B^{v}C^{v}-A^{v}+A^{v}B^{v}-A^{v}B^{v}C^{v}+A^{v}C^{v}
        \end{aligned}\notag
    \end{equation}
    所以命题得证
\end{enumerate}
\noindent


\vspace{10pt}
\noindent
{\bf 题目3:}求下列公式的合取范式和析取范式
\begin{enumerate}
    \item[(1)] $\neg(q\rightarrow p) \wedge (r \rightarrow \neg s)$
    \item[(3)] $\neg(p\vee q) \leftrightarrow (p \wedge q)$
\end{enumerate}


\vspace{5pt}
\noindent
{\bf 答:}
\begin{enumerate}
    \item[(1)] 合取范式:
    \begin{align*}
        \neg(q\rightarrow p) \wedge (r \rightarrow \neg s) &\Leftrightarrow \neg(\neg q\vee p) \wedge (\neg r\vee \neg s)\\
        &\Leftrightarrow q \wedge \neg p \wedge (\neg r \vee \neg s)
    \end{align*}析取范式:
    \begin{align*}
        \neg(q\rightarrow p) \wedge (r \rightarrow \neg s) &\Leftrightarrow \neg(\neg q\vee p) \wedge (\neg r\vee \neg s)\\
        &\Leftrightarrow (q \wedge \neg p) \wedge (\neg r \vee \neg s)\\
        &\Leftrightarrow (q \wedge \neg p \wedge \neg r) \vee (q \wedge \neg p \wedge \neg s)
    \end{align*}
    \item[(3)] 合取范式:
    \begin{align*}
        \neg(p\vee q) \leftrightarrow (p \wedge q) &\Leftrightarrow (\neg(p\vee q) \rightarrow (p \wedge q)) \wedge((p \wedge q) \rightarrow \neg(p\vee q))\\
        &\Leftrightarrow ((p\vee q) \vee (p \wedge q)) \wedge (\neg(p \wedge q) \vee \neg(p\vee q))\\
        &\Leftrightarrow (p\vee q)\wedge (\neg p\vee \neg q)
    \end{align*}析取范式:
    \begin{align*}
        \neg(p\vee q) \leftrightarrow (p \wedge q) &\Leftrightarrow (\neg(p\vee q) \rightarrow (p \wedge q)) \wedge((p \wedge q) \rightarrow \neg(p\vee q))\\
        &\Leftrightarrow ((p\vee q) \vee (p \wedge q)) \wedge (\neg(p \wedge q) \vee \neg(p\vee q))\\
        &\Leftrightarrow (p\vee q)\wedge \neg(p\wedge q)\\
        &\Leftrightarrow (p\wedge \neg q)\vee (q\wedge \neg p)
    \end{align*}
    \end{enumerate}
\noindent

\vspace{10pt}
\noindent
{\bf 题目4:}求下列公式的主合取范式和主析取范式
\begin{enumerate}
\item[(1)] $p\rightarrow p \wedge q$
\item[(3)] $(p\rightarrow p\wedge q)\vee r$
\end{enumerate}

\vspace{5pt}
\noindent
{\bf 答:}
\begin{enumerate}
\item[(1)] 主合取范式:
\begin{align*}
    p\rightarrow p \wedge q &\Leftrightarrow \neg p \vee(p \wedge q)\\
    &\Leftrightarrow \neg p\vee q
\end{align*}主析取范式:
\begin{align*}
    p\rightarrow p \wedge q &\Leftrightarrow \neg p \vee(p \wedge q)\\
    &\Leftrightarrow (\neg p\wedge(q\vee \neg q))\vee(p\wedge q)\\
    &\Leftrightarrow (\neg p\wedge q)\vee(\neg p\wedge \neg q)\vee(p\wedge q)
\end{align*}
\item[(3)] 主合取范式:
\begin{align*}
    (p\rightarrow p\wedge q)\vee r &\Leftrightarrow (\neg p \vee (p\wedge q))\vee r\\
    &\Leftrightarrow \neg p\vee q \vee r
\end{align*}主析取范式:
\begin{align*}
    &(p\rightarrow p\wedge q)\vee r\\
    \Leftrightarrow &\neg p \vee (p\wedge q)\vee r\\
    \Leftrightarrow &(\neg p\wedge(\neg q\vee q)\wedge(\neg r\vee r))\vee ((p\wedge q)\wedge(\neg r\vee r)) \vee (r\wedge(\neg q\vee q)\wedge(\neg p\vee p))\\
    \Leftrightarrow &(\neg p\wedge q\wedge r)\vee(\neg p\wedge q\wedge \neg r)\vee(\neg p\wedge \neg q\wedge r)\vee(\neg p\wedge \neg q\wedge \neg r)\vee(p\wedge q\wedge r)\vee(p\wedge q\wedge \neg r)\vee(p\wedge \neg q\wedge r)
\end{align*}
\end{enumerate}
\noindent

\vspace{10pt}
\noindent
{\bf 题目5:}用$\{\neg, \rightarrow\}$表示下列公式
\begin{enumerate}
\item[(1)] $p\vee (p\wedge q) \leftrightarrow p$
\item[(3)] $((p\wedge q)\wedge r) \leftrightarrow (p \wedge (q\wedge r))$
\item[(5)] $(p\vee (q \wedge r)) \leftrightarrow (p \vee q) \wedge (p \vee r)$
\end{enumerate}

\vspace{5pt}
\noindent
{\bf 答:}
\noindent


\vspace{10pt}
\noindent
{\bf 题目6:}用$\{\uparrow, \downarrow\}$表示下列公式
\begin{enumerate}
\item[(1)] $\neg p \vee q$
\item[(3)] $\neg p \vee \neg q$
\end{enumerate}

\vspace{5pt}
\noindent
{\bf 答:}
\noindent

\end{document}
