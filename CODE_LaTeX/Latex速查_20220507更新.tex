\documentclass{article}
\usepackage{geometry}
\geometry{a4paper,left=1cm,right=1cm,top=1.5cm,bottom=1.2cm}

\usepackage{ctex}
\usepackage{amsmath}
\usepackage{yhmath}
\usepackage{amssymb}
\usepackage{extarrows}
\usepackage{enumerate}
\usepackage{makecell} % 表格内换行
\usepackage{paralist}
\usepackage{fancyhdr}
\usepackage{datetime} 
\usepackage{graphicx}
\usepackage{wrapfig}
\usepackage{fontspec}
\usepackage{bm}       % 同时黑体斜体
\usepackage{listings} % 插入代码
\usepackage[all]{xy}
\usepackage{esint}
\usepackage{bigints}
\usepackage{mathrsfs}
\usepackage{tcolorbox}
\usepackage{ulem}
\usepackage{tikz}
\usepackage{fontawesome5}
\usepackage{multitoc} %双栏
\usepackage[hang,flushmargin,perpage]{footmisc}

%\setcounter{tocdepth}{1}  % 设置目录深度

\columnseprule 1pt
%\columnsep 12pt  

\special{dvipdfmx:config z 0}

% 以下3行用于设定中文字体,若无对应字体文件,则无法编译
%\setCJKmainfont[BoldFont=SourceHanSansCN-Bold]{SourceHanSerifCN-Regular}
%%\setCJKmainfont[BoldFont=OPPOSans-B]{SourceHanSerifCN-Regular}
%\newCJKfontfamily\KaiTifont{gkai00mp.ttf}

\newcommand{\q}{\quad}
\newcommand*{\mycircled}[1]{\lower.7ex\hbox{\tikz\draw (0pt, 0pt)%
circle (.4em) node {\makebox[0.5em][c]{\small #1}};}}

\renewcommand{\today}{\number\year-\number\month-\number\day}

\newcommand\parallelogram
{\mathord{\text{
\tikz[baseline] 
\draw (0em, .1ex)   -- ++(0.8em, 0ex) 
-- ++(0.2em, 1.2ex) -- ++(-0.8em, 0ex)
-- cycle;} } }

\newcommand{\myfootnote}[1]{
    \renewcommand{\thefootnote}{}
    \footnotetext{\scriptsize#1}
    \renewcommand{\thefootnote}{\arabic{footnote}}
%    \renewcommand{\thefootnote}{\ding{\numexpr171+\value{footnote}}}
}

\usepackage{draftwatermark, everypage}
\SetWatermarkText{磁悬浮青蛙呱呱呱,水印}
\SetWatermarkLightness{0}
\SetWatermarkAngle{80}
\SetWatermarkColor{gray}
\SetWatermarkScale{0.06}%设置水印的显示大小

\lstset
{
    language=[LaTeX]TeX,
    backgroundcolor = \color{gray!20},
%    breaklines=true,
%    basicstyle=\tt\small,
    basicstyle=\tt\normalsize,
%    keywordstyle=\color{blue},
%    identifierstyle=\color{magenta},
    aboveskip=0pt,
    belowskip=0pt,
}

\pagestyle{fancy}
\rhead{右页眉}
\lhead{左页眉}
\chead{中页眉}
\rfoot{右页脚}
\lfoot{左页脚,修订日期\today}
\cfoot{中页脚,第\thepage 页} 

\allowdisplaybreaks

\AtBeginDocument{
    \addtolength{\abovedisplayskip}{-2mm}
    \addtolength{\abovedisplayshortskip}{-2mm}
    \addtolength{\belowdisplayskip}{-2mm}
    \addtolength{\belowdisplayshortskip}{-2mm} }   

\begin{document}
\begin{center}
{\Large \textbf{\LaTeX 简明速查手册} }
\end{center}  

%\tableofcontents

\begin{multicols}{2}     
\begin{enumerate}

\item \LaTeX 中的\verb|\usepackage{amsmath}|类似于C语言的\\ \verb|#include<stdio.h>|
和Python的\verb|import numpy|,\\ 常用\verb|usepackage|如下:
\begin{lstlisting}
amsmath,amssymb,bm,ctex,datetime, 
diagbox,enumerate,esint,extarrows,
fancyhdr,fontspec,geometry,graphicx,
listings,longtable,makecell,multicol,
tabularx,tcolorbox,tikz,xcolor  
\end{lstlisting} 
其中\verb|ctex|包用于提供中文显示。

\item \textbf{上标}\q \verb|^|\quad \textbf{下标}\q \verb|_| \\
$ A_b^c $\q \verb|A_b^c| \q\q\q\q 
$ A_{bc}^{def} $\q \verb|A_{bc}^{def}| \\
如果上下标的字母不止一个,则需要加大括号。\\
组合数$ \binom{n}{2} $:\verb|\binom{n}{2}|或\verb|{n\choose 2}|

\item \textbf{分数与根号}:\\
高度自适应的分数(在行内较矮,行间较高):\verb|\frac{}{}|\\
强制较高的分数$ \dfrac{\pi^2}{6} $:\verb|\dfrac{\pi^2}{6}| \\
强制较矮的分数$ \tfrac{\pi^2}{6} $:\verb|\tfrac{\pi^2}{6}| \\
$ \sqrt{5} $\q \verb|\sqrt{5}| \q\q\q\q 
$ \sqrt[3]{5} $\q \verb|\sqrt[3]{5}| 

\item \textbf{运算符}
\vspace{-5mm}
\begin{multicols}{2} 
$ + $\q \verb|+| \\
$ \oplus $\q \verb|\oplus|\\
$ \times $\q \verb|\times| \\
$ \otimes $\q \verb|\otimes| \\
$ \div $\q \verb|\div| \\
$ \neq $\q \verb|\neq| \\
$ \leq $\q \verb|\leq| \\
$ \leqslant $\q \verb|\leqslant| {\footnotesize 需\verb|{amssymb}|} \\
$ \geq $\q \verb|\geq| \\
$ \geqslant $\q \verb|\geqslant| {\footnotesize 需\verb|{amssymb}|} \\
$ \equiv $\q \verb|\equiv| \\
$ \sim $\q \verb|\sim| \\
$ \approx $\q \verb|\approx| \\
$ \cong $\q \verb|\cong| \\
$ \pm $\q \verb|\pm| \\
$ \mp $\q \verb|\mp| \\
$ \in $\q \verb|\in| \\
$ \cap $\q \verb|\cap,\bigcap| \\
$ \cup $\q \verb|\cup,\bigcup| \\
$ \wedge $\q \verb|\wedge,\bigwedge| \\
$ \vee $\q \verb|\vee,\bigvee| \\
$ \subset $\q \verb|\subset| \\
$ \supset $\q \verb|\supset| \\
$ \subseteq $\q \verb|\subseteq| \\
$ \supseteq $\q \verb|\supseteq| \\
$ \int $\q \verb|\int| \\
$ \iint $\q \verb|\iint| \\
$ \iiint $\q \verb|\iiint| \\
$ \iiiint $\q \verb|\iiiint| \\
$ \oint $\q \verb|\oint|  
\end{multicols}
\vspace{-5mm}
$ \varoiint $\q \verb|\varoiint| 需\verb|{esint}| \\
$ \ointctrclockwise $\q \verb|\ointctrclockwise| 需\verb|{esint}| \\
$ \varointclockwise $\q \verb|\varointclockwise| 需\verb|{esint}| \\
(“需\verb|{esint}|”是指需要\verb|\usepackage{esint}|) \\
修改不等号的样式:
\begin{lstlisting}
\renewcommand\leq\leqslant
\renewcommand\geq\geqslant    
\end{lstlisting} 
修改不等号样式以后,临时想使用原始样式的不等号,
则需采用以下曲折方法,在\verb|\renewcommand\leq···|\\
之前加上以下两行
\begin{lstlisting}
\let\leqstore\leq
\let\geqstore\geq    
\end{lstlisting} 
即用两个新的命令\verb|\leqstore|,\verb|\geqstore|保存不等号的旧样式,
想用旧样式时,使用\verb|\leqstore|,\verb|\geqstore|即可。

\item 特殊符号(转义)显示:\\
@符号无需转义,可直接显示
\vspace{-5mm}
\begin{multicols}{2} 
$ \$ $\q \verb|\$| \\
$ \# $\q \verb|\#| \\
$ \% $\q \verb|\%| \\
$ \{ $\q \verb|\{| \\
$ \} $\q \verb|\}| \\
$ \& $\q \verb|\&| \\
$ \^{} $\q \verb|\^{}| \\
$ \_{} $\q \verb|\_{}| \\
$ \~{} $\q \verb|\~{}|或\verb|\sim| 
\end{multicols} 
\vspace{-5mm} 
\textbackslash \q \verb|$\backslash$| 或 \verb|\textbackslash| \\
还有一个统一的方法,就是将特殊符号置于\verb|\verb|| |的两条竖线之内。

\item \textbf{其它符号与形状}:
\vspace{-5mm}
\begin{multicols}{2} 
    $ \cdot $\q \verb|\cdot| \\
    $ \cdots $\q \verb|\cdots| \\
    $ \vdots $\q \verb|\vdots| \\
    $ \ddots $\q \verb|\ddots| \\
    $ \odot $\q \verb|\odot| \\
    $ \hbar $\q \verb|\hbar| \\
    $ \infty $\q \verb|\infty| \\
    $ \partial $\q \verb|\partial| \\
    $ \varnothing $\q \verb|\varnothing| \\
    $ \Delta $\q \verb|\Delta| \\
    $ \square $\q \verb|\square|,\verb|\Box|  \\
    $ \circ $\q \verb|\circ| \\
    $ \angle $\q \verb|\angle| \\
    $ \parallelogram $\q \verb|\parallelogram|
\end{multicols} 
\vspace{-5mm} 
(要先输入下方代码,\verb|\parallelogram|命令才能生效)。
\begin{lstlisting}
\usepackage{tikz}
\newcommand\parallelogram
{\mathord{\text{
\tikz[baseline] 
\draw (0em, .1ex) -- ++(0.8em, 0ex) 
-- ++(0.2em, 1.2ex) -- ++(-0.8em, 0ex)
-- cycle;} } }
\end{lstlisting} 

\item \textbf{极限}、\textbf{连加}、\textbf{连乘}、\textbf{积分}:\\
$ \lim_{n\to\infty} $\q \verb|\lim_{n\to\infty}| \\
$ \varlimsup_{n\to\infty} $\q \verb|\varlimsup_{n\to\infty}| \\
$ \varliminf_{n\to\infty} $\q \verb|\varliminf_{n\to\infty}| \\
$ \sum_{n=1}^{\infty} $\q\q \verb|\sum_{n=1}^{\infty}| \\
$ \prod_{n=1}^{\infty} $\q\q \verb|\prod_{n=1}^{\infty}| \\
$ \int_{0}^{+\infty} $\q\q\q \verb|\int_{0}^{+\infty}| \\
以上代码在行内公式中效果如上,而在行间公式中的效果如下:
\begin{gather*}
    \lim_{n\to\infty},\q \sum_{n=1}^{\infty},\q 
    \prod_{n=1}^{\infty},\q \int_{0}^{+\infty}
\end{gather*}
如果要在行内显示跟行间一样的效果,则加上\verb|\limits|或
\verb|{\displaystyle}|,如下:
\begin{lstlisting}
\lim\limits_{n\to\infty}
\sum\limits_{n=1}^{\infty}
\prod\limits_{n=1}^{\infty}

{ \displaystyle \int_{0}^{+\infty} }
\end{lstlisting} 
可以为\verb|\lim\limits_{}|, \verb|\sum\limits_{}^{}|,\\
\verb|\prod\limits_{}^{}|, \verb|\displaystyle|等常用代码
指定快捷键,提高效率。如果在\verb|\begin{document}|之前加上一句
\verb|\everymath{\displaystyle}|,那么所有行内公式按行间样式显示,
\verb|\lim,\sum,\prod|无需加\verb|\limits|,而\verb|\int|无需加
\verb|\displaystyle|,但副作用是会让行内的连加号、连乘号十分巨大,
对比如下$ \sum\limits_{n=1}^{\infty},\ 
\prod\limits_{n=1}^{\infty},\ { \displaystyle 
\sum_{n=1}^{\infty},\ \prod_{n=1}^{\infty} } $.

虽然
\begin{lstlisting}
\usepackage{bigints}
\end{lstlisting} 
后可以用命令
\begin{lstlisting}
\bigintssss,\bigintsss,\bigintss,
\bigints,   \bigint    
\end{lstlisting} 
获得不同大小的积分号(而无需\verb|\displaystyle|),
但这个系列的积分号太粗了,不美观。\\
比如\verb|\bigintss|:$ \bigintss_{0}^{+\infty} \sin(x^2){\rm d}x
=\dfrac{\sqrt{2\pi}}{4} $.\\
\\
\textbf{多重极限}:\\
$ \lim\limits_{x\to x_0 \atop y\to y_0} $\q 
\verb|\lim\limits_{x\to x_0 \atop y\to y_0}| \\
$ \lim\limits_{\substack{w\to w_0\\ x\to x_0\\ y\to y_0\\ z\to z_0}} $  
\begin{lstlisting}
\lim\limits_{\substack{w\to w_0\\ 
     x\to x_0\\ y\to y_0\\ z\to z_0}}    
\end{lstlisting} 

\item \textbf{括号}:
\begin{lstlisting}
\left(    \right),  \left[   \right]
\left\{  \right\},  \left|   \right|
\big,\bigg,\bigl,\bigr,\biggl,\biggr 
\Big,\Bigg,\Bigl,\Bigr,\Biggl,\Biggr
\end{lstlisting} 
直接使用(),[],\{\},括号的高度不会随着括号中的内容高度而变化,
比如$ (\dfrac{3}{4})^2,[\dfrac{\pi^2}{6}],\{\dfrac{\pi^4}{90}\} $.\\
使用\verb|\left( \right)|,则能让括号随内容增高而变高,
比如$ \left(\dfrac{\pi^2}{6}\right)^2 $. \\
使用\verb|\bigg( \bigg)|,\verb|\Bigg( \Bigg)|之类的,不同的命令代表不同尺寸的括号,而与括号中的内容无关。使用\verb|\left \right|时,内部不能出现换行符\verb|\\|,
若需要较高的括号,就要改用\verb|\bigg( \bigg)|等。\\ 
如果只需要显示单侧的括号(最常见的情形是只显示左大括号),现以左侧为例,\verb|\right|不能省略,必须与\verb|\left|配对出现,配对方法是要把右括号改成小数点,即输入\verb|\right.|,比如$ \left\{\dfrac{\pi^2}{6}\right. $的代码是
\begin{lstlisting}
\left\{\dfrac{\pi^2}{6}\right.    
\end{lstlisting} 
而\verb|\big|系列的括号可以直接省去一侧,无需配对出现。

$ \left\| \dfrac{a}{b} \right\| $\q 
\verb|\left\||\verb|\dfrac{a}{b}\right\|| \\
$ \left\langle \dfrac{a}{b} \right\rangle $\q 
\verb|\left\langle\dfrac{a}{b}\right\rangle| \\
$ \left\lfloor \dfrac{a}{b} \right\rfloor $\q 
\verb|\left\lfloor\dfrac{a}{b}\right\rfloor| \\
$ \left\lceil  \dfrac{a}{b} \right\rceil $\q 
\verb|\left\lceil \dfrac{a}{b}\right\rceil | 


\item 行内公式$ a^2+b^2=c^2 $使用\verb|$ a^2+b^2=c^2 $|即可。\\
行间公式可用语法很多,比如\verb|\[  \], $$  $$|,
这两种环境只能输入单行公式,换行符\verb|\\|在其中无效。
行间公式还可以用
\verb|\begin{xx} \end{xx}|之类,其中\verb|xx|可以是
\begin{lstlisting}
    align(*),alignat(*),flalign(*)
  equation(*), gather(*),multline(*)
\end{lstlisting} 
带*的环境不给公式编号,不带*的环境自动给公式编号,
使用\verb|\notag|或\verb|\nonumber|可隐藏任意一行公式的编号。
\verb|equation(*)|也只能输入单行公式,换行符\verb|\\|在其中无效,
但在其中嵌入\verb|split|环境后就能输入多行公式了,
好处是多行公式只有一个编号。
\begin{lstlisting}
\begin{equation} \label{aaa1}
    \begin{split}
         &\ x^4+2x^3+11x^2+18x+18 \\
        =&\ (x^2+2x+2)(x^2+9)  \\
        =&\ (x^2+x+3)^2+(2x+3)^2
    \end{split}
\end{equation}    
\end{lstlisting} 

\begin{equation} \label{aaa1}
    \boxed{
    \begin{split}
        &\ x^4+2x^3+11x^2+18x+18 \\
        =&\ (x^2+2x+2)(x^2+9)  \\
        =&\ (x^2+x+3)^2+(2x+3)^2
    \end{split} }
\end{equation} 

用\verb|\label{aaa1}|给公式加标签,然后用\verb|\ref{aaa1}|引用公式(的编号),
\verb|\pageref{aaa1}|引用公式所在的页码。\\
\\
\verb|alignat|和\verb|align|环境区别如下(不明显,\verb|align|整体稍微宽一点):
\begin{lstlisting}
\begin{alignat*}{3}
    2x+3 &= 5678y-8765z &+ 20 \\
      4x &= y+z &+ 11112222 
\end{alignat*}    
\end{lstlisting} 
\tcbset{notitle,before={\noindent},after={\noindent},
    colback=white,top={1mm},bottom={1mm},}
\begin{tcolorbox}
    \vspace{-5mm}
\begin{alignat*}{3}   
    2x+3 &= 5678y-8765z &+ 20 \\
    4x &= y+z &+ 11112222  
\end{alignat*} 
\end{tcolorbox}
\begin{lstlisting}
\begin{align*}
    2x+3 &= 5678y-8765z &+ 20 \\
      5x &= y+z &+ 33334444 
\end{align*}    
\end{lstlisting} 
\begin{tcolorbox}
    \vspace{-3mm}
\begin{align*}
    \begin{aligned}
      2x+3 &= 5678y-8765z &+ 20 \\
        5x &= y+z &+ 33334444 
    \end{aligned} 
\end{align*}
\end{tcolorbox}
\verb|gather(*)|环境中不能出现对齐符号\verb|&|,否则报错。
此环境下所有行的公式全部居中对齐。
\begin{lstlisting}
\begin{gather*}
    2x+3 = 5678y-8765z + 20 \\
      6x = y+z + 55556666 
\end{gather*}    
\end{lstlisting} 
\begin{gather*}
    \boxed{
    \begin{gathered}
        2x+3 = 5678y-8765z + 20 \\
        6x = y+z + 55556666 
    \end{gathered} }
\end{gather*}
\verb|cases|环境对于带左大括号的情形特别有用,比如分段函数、方程联立等,
\begin{lstlisting}
\begin{align*}
    \begin{cases}
        2x+3y=7 \\
        3x+5y=8
    \end{cases}
\end{align*}    
\end{lstlisting} 
\begin{align*}
    \boxed{
    \begin{cases}
        2x+3y=7 \\
        3x+5y=8
    \end{cases} }
\end{align*}
虽然用
\begin{lstlisting}
\begin{align*}
    \left\{  
        \begin{aligned}
            & 2x+3y=7 \\
            & 3x+5y=8
        \end{aligned}  
    \right.
\end{align*}    
\end{lstlisting} 
也能实现同样效果,但显然是\verb|cases|更方便。\\
\verb|multline(*)|环境第一行左对齐,中间的行居中对齐,最后一行右对齐,用得较少。
\begin{lstlisting}
\begin{multline}
    1-line \\
    2-line \\
    3-line \\
    4-line 
\end{multline}    
\end{lstlisting}
\begin{tcolorbox} 
    \vspace{-5mm}
    \begin{multline}
        1-line \\
        2-line \\
        3-line \\
        4-line 
    \end{multline}
\end{tcolorbox}
公式环境中要加汉字,则必须置于\verb|\text{}|之内。\\
实现文本居中对齐使用\verb|center|环境
\begin{lstlisting}
\begin{center}

\end{center}
\end{lstlisting} 
以上给公式外围加边框用的是:
\begin{lstlisting}
\begin{align*}
    \boxed{
        \begin{aligned}
            ······
        \end{aligned} 
    }
\end{align*}
\end{lstlisting}
或
\begin{lstlisting}
\usepackage{tcolorbox}
\tcbset{before={\noindent},
    after={\noindent},colback=white}
\begin{tcolorbox}
    \vspace{-5mm}
    \begin{align*}
        ······
    \end{align*} 
\end{tcolorbox}
\end{lstlisting} 

想让公式编号带上“章”序号或“节”序号,可使用
\begin{lstlisting}
\numberwithin{equation}{chapter}
\numberwithin{equation}{section}    
\end{lstlisting} 


\item \textbf{矩阵和行列式}:\\
$ \begin{pmatrix}
   a_{11} & a_{12}  \\
   a_{21} & a_{22}  \\
\end{pmatrix} $
\begin{lstlisting}
\begin{pmatrix}
    a_{11} & a_{12}  \\
    a_{21} & a_{22}  \\
\end{pmatrix}    
\end{lstlisting} 
$ \begin{bmatrix}
    a_{11} & a_{12}  \\
    a_{21} & a_{22}  \\
\end{bmatrix} $ 用\, \verb|bmatrix|,\ 
$ \begin{vmatrix}
    a_{11} & a_{12}  \\
    a_{21} & a_{22}  \\
\end{vmatrix} $ 用\, \verb|vmatrix| \\
不带括号和竖线用\verb|matrix|,大括号用\verb|Bmatrix|,\\
双竖线用\verb|Vmatrix|.\\
三种省略号:
$ \cdots $\  \verb|\cdots|,\ \  
$ \vdots $\  \verb|\vdots|,\ \
$ \ddots $\  \verb|\ddots|

\item \textbf{函数}:
\begin{lstlisting}
\arg,\exp, \inf,\sup,   \max,\min
\sin,\sinh,\arcsin,\cos,\cosh,\arccos
\tan,\tanh,\arctan
\log,\ln,\lg,      \deg,\det,\dim     
\end{lstlisting} 
%\vspace{-3mm}
这些函数只能在公式环境中使用,而且字体是正体,
如果不在前面加$ \backslash $,直接输入$ sin,cos,log $,
字体就是斜体。

\item 公式中,某些特殊含义的字母需要用正体而非斜体,比如自然对数底数e,虚数单位i和微分符号d,有两种方法,分别是\verb|\mathrm{e}|(推荐)和\verb|{\rm e}|(不推荐),比如
\begin{lstlisting}
{\rm e}^{{\rm i}\theta}=
    \cos\theta+{\rm i}\sin\theta \\
\int_0^{+\infty}\frac{x}{\mathrm{e}^x
    -1}\mathrm{d}x=\frac{\pi^2}{6}    
\end{lstlisting} 
%\vspace{-5mm}
\begin{gather*}
    {\rm e}^{{\rm i}\theta}=\cos\theta+{\rm i}\sin\theta \\
    \int_{0}^{+\infty}\dfrac{x}{\mathrm{e}^x
    -1}\mathrm{d}x=\dfrac{\pi^2}{6}
\end{gather*}
公式环境下e、i、d都不用正体的效果是:$ e,i,d $.

\item 自定义新的命令:\verb|\newcommand{}{}|,
效果类似于C语言的宏替换\verb|#define|. 
比如嫌\verb|\quad|太麻烦,可以先\verb|\newcommand{\q}{\quad}|,
然后就能用\verb|\q|代替\verb|\quad|. 
在\verb|\newcommand{\im}{{\rm i}}|之后,就能用\verb|\im|实现
正体的虚数单位$ {\rm i} $.因为\verb|\i|已经在某个包中定义过了,
所以也可以用\verb|\renewcommand{\i}{{\rm i}}|覆盖掉\verb|\i|的定义。
对e和d可类似处理,提高输入效率。

\item \textbf{希腊字母}:
\vspace{2mm} \\
\begin{tabular}{|l|l|l|l|l|l|}
    \hline
$ \alpha $ & \verb|\alpha| & $ \beta $ & \verb|\beta| & $ \gamma $ & \verb|\gamma|  \\ \hline
$ \delta $ & \verb|\delta| & $ \epsilon $ & \verb|\epsilon| & $ \varepsilon $ & \verb|\varepsilon| \\ \hline
$ \zeta  $ & \verb|\zeta|  & $ \eta $ & \verb|\eta| & $ \theta $ & \verb|\theta| \\ \hline
$ \lambda$ & \verb|\lambda|& $ \mu  $ & \verb|\mu| & $ \nu $ & \verb|\nu|  \\ \hline
$ \xi    $ & \verb|\xi|    & $ \pi $ & \verb|\pi| & $ \rho $ & \verb|\rho| \\ \hline
$ \sigma $ & \verb|\sigma| & $ \tau $ & \verb|\tau| & $ \phi $ & \verb|\phi| \\ \hline
$ \varphi$ & \verb|\varphi|& $ \psi $ & \verb|\psi| & $ \omega $ & \verb|\omega| \\ \hline
\end{tabular} 
\vspace{2mm}  \\
以下字母存在大写形式(省略了一些带\verb|\var|前缀的),只需把首字母大写即可。
\begin{lstlisting}
\Gamma,\Delta,\Theta,\Lambda,\Xi,\O,
\Pi,\Sigma,\Upsilon,\Phi,\Psi,\Omega    
\end{lstlisting} 

\item 字母上下加符号:
\begin{multicols}{2}
$ \overline{a} $\q \verb|\overline{a}| \\
$ \underline{a} $\q \verb|\underline{a}| \\
$ \overbrace{a} $\q \verb|\overbrace{a}| \\
$ \underbrace{a} $\q \verb|\underbrace{a}| \\
$ \overleftarrow{a} $\q \verb|\overleftarrow{a}| \\
$ \overrightarrow{a} $\q \verb|\overrightarrow{a}| \\
$ \stackrel{b}{a} $\q \verb|\stackrel{b}{a}| \\
$ \overset{b}{a} $\q \verb|\overset{b}{a}| \\
$ \underset{b}{a} $\q \verb|\underset{b}{a}| \\
$ \acute{a} $\q \verb|\acute{a}| \\
$ \grave{a} $\q \verb|\grave{a}| \\
$ \tilde{a} $\q \verb|\tilde{a}| \\
$ \widetilde{abc} $\q \verb|\widetilde{abc}| \\
$ \bar{a} $\q \verb|\bar{a}| \\
$ \vec{a} $\q \verb|\vec{a}| \\
$ \hat{a} $\q \verb|\hat{a}| \\
$ \widehat{abc} $\q \verb|\widehat{abc}| \\
$ \check{a} $\q \verb|\check{a}| \\
$ \breve{a} $\q \verb|\breve{a}| \\
$ \dot{a} $\q \verb|\dot{a}| \\
$ \ddot{a} $\q \verb|\ddot{a}| \\
$ \dddot{a} $\q \verb|\dddot{a}| 
\end{multicols}

\item \textbf{中文加下划线}:(需\verb|\usepackage{ulem}|)
\vspace{-5mm}
\begin{multicols}{2}
\uline{单下划线}\q \verb|\uline{}| \\
\uuline{双下划线}\q \verb|\uuline{}| \\
\uwave{波浪线}\q\q \verb|\uwave{}| \\
\sout{删除线}\q\q \verb|\sout{}| \\
\dashuline{虚下划线}\q \verb|\dashuline{}| \\
\dotuline{点下划线}\q \verb|\dotuline{}| 
\end{multicols}

\item \textbf{箭头}:
\vspace{-5mm}
\begin{multicols}{2}
$ \to $\q \verb|\to| \\
$ \rightarrow $\q \verb|\rightarrow| \\
$ \Rightarrow $\q \verb|\Rightarrow| \\
$ \longrightarrow $\q \verb|\longrightarrow| \\
$ \leftarrow $\q \verb|\leftarrow| \\
$ \Leftarrow $\q \verb|\Leftarrow| \\
$ \uparrow $\q \verb|\uparrow| \\
$ \downarrow $\q \verb|\downarrow| 
\end{multicols}
\vspace{-5mm}
$ \xrightarrow[a,b]{c,d} $\q \verb|\xrightarrow[a,b]{c,d}| \\
$ \xlongequal[140^{\circ}{\rm C}]{\text{稀硫酸}} $\q 
 (\verb|\xlongequal|需\verb|\usepackage{extarrows}|) \\  
\verb|\xlongequal[140^{\circ}{\rm C}]{\text{稀硫酸}}|


\item \textbf{插入表格}:
\begin{lstlisting}
\begin{tabular}{|c|c|}
    \hline
    &  \\
    \hline
    &  \\
    \hline
\end{tabular}    
\end{lstlisting} 
表格内换行:
\begin{lstlisting}
\usepackage{makecell}  
\makecell[l]{第一行 \\ 第二行 \\ ···}  
\end{lstlisting} 
合并单元格则使用\verb|\multicolumn|和\verb|\multirow|.\\
跨页的长表格使用\verb|\begin{longtable}···|. \\
表格行距控制:\verb|\renewcommand{\arraystretch}{1.5}| 


\item \textbf{插入图片}:
\begin{lstlisting}
\usepackage{graphicx}    
\begin{figure}
\centering
\includegraphics[width=
        0.3\linewidth]{图片名}
\caption{图片标题}
\label{xxx1}
\end{figure} 
\end{lstlisting} 
位置控制:\verb|h t b p ! H| \\
四种宽度:\\
\verb|\linewidth|\q\q 当前行的宽度 \\
\verb|\columnwidth|\q 当前分栏的宽度 \\
\verb|\textwidth|\q\q\ 整个页面版芯的宽度 \\
\verb|\paperwidth|\q\q 整个页面纸张的宽度


\item \textbf{添加页眉页脚}:
\begin{lstlisting}
\usepackage{fancyhdr}
\pagestyle{fancy}
\lhead{左页眉}
\chead{中页眉}
\rhead{右页眉}
\lfoot{左页脚,修订日期\today}
\cfoot{中页脚,第\thepage 页}
\rfoot{右页脚}
\end{lstlisting} 

\item \textbf{添加水印}:\\
使用\verb|{xwatermark}|包会遇到报错\\
“\verb|Extra \endgroup. \begin{document}|”;\\
\verb|{background}|第一页水印的颜色比后面的页更深,第二页水印内容也有异常;
\verb|{watermark}|(2004)和\verb|{draftmark}|(2009)太旧,均无法使用。\\
下面给出\verb|{draftwatermark}|用法示例,但这个包有时会出现水印文字
重叠到一起的问题(本文档编译时经常遇到这个问题,但不是100\% 出现)。
\begin{lstlisting}
\usepackage{draftwatermark}
\usepackage{everypage}
\SetWatermarkText{磁悬浮青蛙呱呱呱,水印}
\SetWatermarkLightness{0}
\SetWatermarkAngle{80}
\SetWatermarkColor{gray}
\SetWatermarkScale{0.07}
\end{lstlisting} 

\item 添加带编号\textbf{脚注}\footnote{这是用
    $\backslash$footnote\{\}添加的带编号脚注。}:
\verb|\footnote{}|. \\
无编号脚注:(自定义了\verb|\myfootnote|命令)\myfootnote{这是用
    $\backslash$myfootnote\{\}添加的无编号脚注。}
\begin{lstlisting}
\newcommand{\myfootnote}[1]{
\renewcommand{\thefootnote}{}
\footnotetext{\scriptsize#1}
\renewcommand{\thefootnote}{
    \arabic{footnote}}  }    
\end{lstlisting} 
把脚注编号改为带圈数字:
\begin{lstlisting}
\renewcommand{\thefootnote}{ 
\ding{\numexpr171+\value{footnote}}}

\newcommand{\myfootnote}[1]{
\renewcommand{\thefootnote}{}
\footnotetext{\scriptsize#1}
\renewcommand{\thefootnote}{
\ding{\numexpr171+\value{footnote}}}}    
\end{lstlisting} 


\item \textbf{允许公式跨页}:\verb|\allowdisplaybreaks|

\item \textbf{新增空白页}:
\begin{lstlisting}
\newpage, \clearpage, \cleardoublepage
\end{lstlisting} 

%\item \textbf{日期}:\verb|\today|

\item \textbf{目录}:\verb|\tableofcontents| \\
设置目录深度:\verb|\setcounter{tocdepth}{3}| \\
设置在几级目录前标记序号:\\ 
\verb|\setcounter{secnumdepth}{4}|

\item \textbf{字体大小}控制:
\begin{lstlisting}
\tiny, \scriptsize, \footnotesize
\small, \normalsize
\large, \Large, \LARGE
\huge, \Huge    
\end{lstlisting} 

文本行距控制:\verb|\linespread{1.3}|(必须放在\\
    \verb|\begin{document}|之前) 

\item \textbf{粗体}:\verb|\textbf{}|,使用时如果恰好换行,
在tex源码中让\verb|\textbf{}|处于新一行,则编译后的粗体前面会多一个空格,
解决方案就是不要恰好在\verb|\textbf{}|前面换行。
斜体命令\verb|\textit{}|只对英文有效,对中文无效。
对英文同时斜体和粗体则需\verb|\usepackage{bm}|,
$ \bm{AB} $:\verb|$\bm{AB}$| \\
用以下命令修改字体,需\verb|\usepackage{fontspec}|\\
设置英文字体:\verb|\setmainfont{Microsoft YaHei}|\\
设置C(中文)、J(日文)、K(韩文)的字体:
\begin{lstlisting}
\setCJKmainfont[BoldFont=OPPOSans-B]{
    SourceHanSerifCN-Regular}
\end{lstlisting} 
\textcolor{red}{设置}\textcolor{green}{文本}\textcolor{gray}{颜色}\textcolor{orange}{(textcolor)}:
\begin{lstlisting}
\textcolor{red}{设置}···
\end{lstlisting} 


\item 部分\ \verb|\part{}| \hspace{2cm} 章\ \ \ \ \verb|\chapter{}| \\
节\ \ \ \ \verb|\section{}|  \hspace{1.45cm} 小节\ \verb|\subsection{}|

\item \textbf{带编号列表}:
\begin{lstlisting}
\usepackage{enumerate}
\begin{enumerate}[(1)]
\item 
\item 
\end{enumerate}
\end{lstlisting} 
\textbf{不带编号列表}:
\begin{lstlisting}
\begin{itemize}
\item 
\item 
\end{itemize}
\end{lstlisting} 

\item 常用\textbf{长度单位}:毫米(mm),厘米(cm),点(pt),ex,em

\item \textbf{交换图}:
\begin{displaymath}
\xymatrix{ V \ar[r]^{\bm{\varphi}}
    \ar[d]_{\bm{\eta}_1} & U\ar[d]^{\bm{\eta}_2} \\
    {\mathbb{K}_n} \ar[r]^{\bm{\varphi}_A} & {\mathbb{K}_m} }
\end{displaymath}

\begin{lstlisting}
\usepackage[all]{xy}
\begin{displaymath}
\xymatrix{ 
V \ar[r]^{\bm{\varphi}}
  \ar[d]_{\bm{\eta}_1} 
& U\ar[d]^{\bm{\eta}_2} \\
{\mathbb{K}_n} \ar[r]^{\bm{\varphi}_A} 
& {\mathbb{K}_m} }
\end{displaymath}    
\end{lstlisting} 

\item \textbf{空格与空白}:
\vspace{-5mm}
\begin{multicols}{2} 
    负空格\q \verb|\!|  \\
    窄空格\q \verb|\,|  \\
    中等空格\q \verb|\:|  \\
    宽空格\q \verb|\;|   \\
    词间空格\q \verb|\ | \\
    四倍空格\q \verb|\quad|  \\
    八倍空格\q \verb|\qquad| 
\end{multicols} 
\vspace{-5mm}
注意,“词间空格”的斜杠后有一个看不见的空格。\\
取消首行缩进:\verb|\noindent| \\
水平空白\q \verb|\hspace{|$ \pm $\verb|2cm}| \\
垂直空白\q \verb|\vspace{|$ \pm $\verb|2cm}| \\
缩小行间公式与上下文之间的空白(必须放在\\
\verb|\begin{document}|之前):
\begin{lstlisting}
\AtBeginDocument{
\addtolength{\abovedisplayskip}{-2ex}
\addtolength{
    \abovedisplayshortskip}{-2ex}
\addtolength{\belowdisplayskip}{-2ex}
\addtolength{
    \belowdisplayshortskip}{-2ex} }    
\end{lstlisting} 

\item \textbf{设置页边距}:
\begin{lstlisting}
\usepackage{geometry}
\geometry{a4paper,left=1cm,right=1cm,
          top=1.5cm,bottom=1.5cm}    
\end{lstlisting} 

\item 英文字母几种变体效果如下:\\
\verb|\mathcal{}|(只能用于大写字母,对小写无效)
\begin{gather*}
\boxed{\mathcal{ABCDEFGHIJKLMNOPQRSTUVWXYZ}}
\end{gather*}
\verb|\mathscr{}|(只能用于大写字母,需\verb|{mathrsfs}|)
\begin{gather*}
\boxed{ \begin{gathered}
    \mathscr{ABCDEFGHIJKLM}\\
    \mathscr{NOPQRSTUVWXYZ}
\end{gathered} }
\end{gather*}
\verb|\mathbb{}|(只能用于大写字母,需\verb|{amssymb}|)
\begin{gather*}
    \boxed{ \begin{gathered}
        \mathbb{ABCDEFGHIJKLMNOPQRSTUVWXYZ}
    \end{gathered} }
\end{gather*}
\verb|\mathfrak{}|(同时适用于大小写,需\verb|{amssymb}|)
\begin{gather*}
    \boxed{ \begin{gathered}
            \mathfrak{ABCDEFGHIJKLMNOPQRSTUVWXYZ}\\
            \mathfrak{abcdefghijklmnopqrstuvwxyz}
    \end{gathered} }
\end{gather*}

\item 自定义带圈数字命令\verb|\mycircled{}|:
\begin{lstlisting}
\newcommand*{\mycircled}[1]{\lower.7ex
    \hbox{\tikz\draw (0pt, 0pt) 
    circle (.4em) node {
    \makebox[0.5em][c]{\small #1}};} }    
\end{lstlisting} 

\item 防止ff,fi,ffi,fl变成连体(Ligature):\\
方法一:\\
\verb|f{}f,f{}i,f{}f{}i,f{}l|,效果:f{}f, f{}i, f{}f{}i, f{}l \\
方法二:\\
\verb|f{f},f{i},f{f}i,  f{l}|,效果:f{f}, f{i}, f{f}i, f{l}

\item 本手册使用了多栏环境
\begin{lstlisting}  
\usepackage{multicol}      
\begin{multicols}{2} 

\end{multicols}    
\end{lstlisting} 
以及带编号列表环境\verb|enumerate|,用
\begin{lstlisting}
\columnseprule 1pt    
\end{lstlisting} 
显示中央分隔竖线并控制线宽。用
\begin{lstlisting}
\columnsep 20pt  
\end{lstlisting} 
控制两栏之间的间隔。显示\LaTeX 代码使用了两种方法,
较短的代码使用了\verb|\verb| | |,大片的代码使用了 
\begin{lstlisting}
\usepackage{listings}
\lstset
{ language=[LaTeX]TeX,
  backgroundcolor=\color{gray!20},
  basicstyle=\tt\normalsize,
  aboveskip=0pt,
  belowskip=0pt, }    
\end{lstlisting}
\vspace{-2mm} 
\begin{tcolorbox}[arc={0mm},colback={gray!20},boxrule={0pt},left={0pt}]
\verb|\begin{lstlisting}| \\
\\
\verb|\end{lstlisting}|
\end{tcolorbox}
除了\verb|lstlisting|,也可以使用 \\
\begin{tcolorbox}[arc={0mm},colback={gray!20},boxrule={0pt},left={0pt}]
\verb|\begin{verbatim}| \\
\\
\verb|\end{verbatim}|
\end{tcolorbox}

\item 以下三个网站可以在线写作以及编译\LaTeX :
\begin{lstlisting}
https://www.texpage.com/
https://www.slager.cn/
https://cn.overleaf.com/
\end{lstlisting} 
以下网站可以识别单个手写的\LaTeX 符号,并提供可能的\LaTeX 代码。
\lstset
{   language=[LaTeX]TeX,
    backgroundcolor = \color{gray!20},
    basicstyle=\tt\small,
    aboveskip=1pt,
    belowskip=1pt, }
\begin{lstlisting}
http://detexify.kirelabs.org/classify.html 
\end{lstlisting} 
Mathpix snip软件(Win,MacOS,Linux,IOS,Android均支持)
能识别手写或印刷的数学公式,包括矩阵和表格等,
然后生成完整的\LaTeX 代码,而且准确率很高,值得尝试。

\end{enumerate}  
\end{multicols}

\end{document}
